\documentclass{beamer} %[12pt]
\usepackage{xcolor}
%\usetheme{boadilla}
%\usetheme{malmoe}
%\usetheme{copenhagen}
%\usecolortheme{rose}
\usecolortheme{beaver}
\usepackage{pgf, graphics}
\usepackage{graphicx}
%\usepackage[left=3cm,top=3cm,right=3cm,nohead,nofoot]{geometry}
\usepackage{hyperref}
\usepackage{setspace}
\usepackage[square]{natbib}
\usepackage{amsmath}
\usepackage{amssymb}
\usepackage{verbatim}
\usepackage{color}
\usepackage{fancyvrb}
\usepackage{bbm}

\begin{filecontents}{ref.bib}
\end{filecontents}

%\usetheme{EastLansing}
%\usepackage{natbib}
\bibliographystyle{apalike}
% make bibliography entries smaller
%\renewcommand\bibfont{\scriptsize}
% If you have more than one page of references, you want to tell beamer
% to put the continuation section label from the second slide onwards
\setbeamertemplate{frametitle continuation}[from second]
% Now get rid of all the colours
\setbeamercolor*{bibliography entry title}{fg=black}
\setbeamercolor*{bibliography entry author}{fg=black}
\setbeamercolor*{bibliography entry location}{fg=black}
\setbeamercolor*{bibliography entry note}{fg=black}
% and kill the abominable icon
\setbeamertemplate{bibliography item}{}


\newcommand{\hl}[1]{\colorbox{yellow}{#1}}
\newcommand{\hlblue}[1]{\colorbox{green}{#1}}
\newcommand{\hlblu}[1]{\colorbox{cyan}{#1}}
\newcommand{\hlred}[1]{\colorbox{cyan}{#1}}
\newcommand{\hlre}[1]{\colorbox{pink}{#1}}
\newcommand{\hlgreen}[1]{\colorbox{pink}{#1}}
\newcommand{\hlgree}[1]{\colorbox{green}{#1}}



\DeclareMathOperator*{\argmax}{\arg\!\max}

\DeclareMathOperator*{\argmin}{\arg\!\min}


\newcommand{\specialcell}[2][c]{%
  \begin{tabular}[#1]{@{}c@{}}#2\end{tabular}}



%\setbeamersize{text margin left=.5cm,text margin right=.5cm}
\newenvironment{changemargin}[2]{%
  \begin{list}{}{%
    \setlength{\topsep}{0pt}%
    \setlength{\leftmargin}{#1}%
    \setlength{\rightmargin}{#2}%
    \setlength{\listparindent}{\parindent}%
    \setlength{\itemindent}{\parindent}%
    \setlength{\parsep}{\parskip}%
  }%
  \item[]}{\end{list}}
\setbeamertemplate{navigation symbols}{}%remove navigation symbols
\usepackage{color}
\newcommand{\hilight}[1]{\colorbox{yellow}{#1}}
\setbeamertemplate{footline}[page number]

\begin{document}


\title[dedup]{Today:  Hierarchical Clustering, Base R Graphics}


\author[Samuel L. Ventura]{\\
  \large{Sam Ventura\\36-315\\Today:  Lab Exam Notes, Hierarchical Clustering, \\Base R Graphics}}
\institute[CMU Statistics]{Department of Statistics\\Carnegie Mellon University}
\date{\today}


\begin{frame}
	\maketitle

	
\end{frame}



\begin{frame}\frametitle{Hierarchical Clustering}
	\small
	
	As we will see on the upcoming lab and HW, we can project the data into lower-dim space and visualize the results.  Why might this be a bad idea?
	
	\vskip 0.5 cm
	
	Another option:  Use \textbf{hierarchical linkage clustering}.
	
	\vskip 4.5 cm
	
	\textbf{Single linkage}:  the distance between two groups is the shortest possible distance between two points, one from each group
	
	\vskip 0.5 cm
	
	\textbf{Complete linkage}:  the distance between two groups is the largest possible distance between two points, one from each group
	
	\vskip 10 cm
	
\end{frame}



\begin{frame}\frametitle{Reminder:  Why We Use ggplot()}
	\small
	
	There are many reasons:
	
	\begin{enumerate}
		\item ``Grammar of graphics''
		\item Default graphs / colors / etc are already nice
		\item Can easily change geometry of plot without changing the code
		\item Can save plots -- including portions of plots -- as objects
		\item Easy to perform multivariate exploration (coloring, faceting, etc)
		\item Legends are generated and updated automatically when you change your graphic
		\item Can build plots in layers (e.g. add points, then add 2D density estimate, then add regression lines, error bars, etc)
		\item Documentation is detailed; easy to find help/tutorials online
		\item ``All the cools kids are doing it''
	\end{enumerate}
	
	
\end{frame}




\begin{frame}\frametitle{Base R Graphics}
	\small
	
	Base R graphics are the standard way to create plots in R
	
	\vskip 0.75 cm
	
	Generally, these use the plot() function to plot specific pieces of data
	
	\vskip 0.75 cm
	
	Big difference from ggplot():  plot() requires vectors for each argument, while ggplot() uses data.frames
	
	\vskip 0.75 cm
	
	plot(x = data\$variable1, y = data\$variable2, col = data\$color\_variable, pch = data\$point\_type\_variable)
	
	\vskip 0.75 cm
	
	Naming conventions in base R graphics are often un-intuitive
	
	\vskip 0.75 cm
	
	Some more advanced statistical methods have specific types of plots implemented in base-R graphics; ggplot() is relatively new, so some of these are not yet implemented in ggplot()
	
\end{frame}





\end{document}
